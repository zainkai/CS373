\documentclass[letterpaper,12pt,titlepage,onecolumn]{IEEEtran}

\usepackage{graphicx}                                        
\usepackage{amssymb}                                         
\usepackage{amsmath}                                         
\usepackage{amsthm}                                          

\usepackage{alltt}                                           
\usepackage{float}
\usepackage{color}
\usepackage{url}
\usepackage{listings}

\usepackage{balance}
\usepackage[TABBOTCAP, tight]{subfigure}
\usepackage{enumitem}
\usepackage{pstricks, pst-node}

\usepackage{geometry}
\geometry{textheight=8.5in, textwidth=6in}

\newcommand{\cred}[1]{{\color{red}#1}}
\newcommand{\cblue}[1]{{\color{blue}#1}}

\usepackage{hyperref}
\usepackage{geometry}

\def\name{Kevin Turkington}
\author{\name}
\title{Lab 1}


%% The following metadata will show up in the PDF properties
\hypersetup{
  colorlinks = true,
  urlcolor = black,
  pdfauthor = {\name},
  pdfkeywords = {cs373},
  pdfpagemode = UseNone
}

\begin{document}
\maketitle
\hrulefill

\section{Binary Analysis}
\subsection{evil.exe}
During my analysis with McAfee's FileInsight program I found the Malware delete or create a collection of scheduled tasks to run every 30 minutes through out the day that run the file: \textit{c:/ntldrs/svchest.exe}. upon further research I found the \textit{svchest.exe} is a windows process that hosts multiple windows processes, which could mean the malware is attempting to stop firewalls or anti-virus scanners from running regularly. Additionally a couple different files are being stored within the ntldrs directory which may be dependencies for the malware to function or a seperate piece of malware altogether. Indicating the sample provided \textit{evil.exe} is probably a downloader, intended for the delivery and installation of the actual payload. upon further inspection of the malicious software I found a lot of vba specfic commands indicating the malware is written in visual basic. \par
Files that are being stored in the boot directory:
\begin{itemize}
    \item ntldrs/Isinter.gif
    \item ntldrs/system.yf
    \item ntldrs/svchest.exe
    \item ntldrs/funbots.bat
\end{itemize}


\subsection{tongji2.exe}
During FileInsight of this file, I found a couple Delphi specfic commands and keywords in a string search. Which could mean the \textit{tongji2.exe} is the main payload of the malware. However the only thing that I could descipher from the keywords was that the program was attempting to spawn seperate threads. Which could mean machines are being for computational capabilitys as a whole.

\section{Run time Analysis}
After setting up the recommnded tools from the lab instructions I found during run time the malware immediately created a couple different GET/POST requests to an outside server specfically to the url \textit{timeless888.com}. Then \textit{evil.exe} proceeded to create a collection of files.
while inspecting the windows task scheduler, it seemed to somewhat reflect what was examined in the binary file from \textit{evil.exe}, creating a 30 minute task running \textit{svchest.exe}. Once the exact hour came and the tasks were run the malware created a GET request to timeless888.com for a page called \textit{tong.htm}, however while conducting a system wide search for \textit{tong.htm} I could not find it. Leading to the conclusion that the website is stored in memory and will be served back to the victum in one way or another. 



\section{research}
After examining article from piazaa posted cas donoghue\cite{Analysis26:online}, I found out the malware was a backdoor Trojan with backdoor capabilities with the addition of serving up a fake webpage (\textit{tong.htm}) used for stealing social security numbers from Korean citizens. The payload of the virus was deployed in several stages starting from a simple visual basic script (\textit{evil.exe}), then downloading the dummy website, and finally installing a backdoor (\textit{tongji2.exe}).

  
% bibliography
\nocite{*}%if nothing is referenced it will still show up in refs
\bibliographystyle{plain}
\bibliography{refs}
%end bibliography

\end{document}